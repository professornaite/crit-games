

We define a cyber security game as a game-theoretic model that captures es-
sential characteristics of %resource allocation decision making 
(i.e. system admin-
istrators’ time) 



% We define a transitional justice game as a game-theoretic model that identifies essential characteristics of % resource allocation decision making (i.e., dynamic and changing real-time value of land and the question of land in indigenous and native communities; reparations for African Americans)


This is a two-player, non-cooperative static game between a
system security team, defender D, which defends an organization’s information
assets and data against external or insider adversaries who are modeled as an
omnipresent attacker A. %Our model follows the definitions of the work done by Korzhyk et al. in [11]. 


We have formulated general-sum games that represent our cyber
security environment, and we have proven that the defender’s Nash strat-
egy is also minimax.

% We have formulate general-sum games that represent our 

we also propose
Singular Value Decomposition (SVD) as an efficient technique to com-
pute approximate equilibria in our games. 



For example for a
scenario with 4 targets, when the number of system administrators equals two
then the available schedules are:
S = {< 1, 1, 0, 0 >, < 1, 0, 1, 0 >, < 1, 0, 0, 1 >,
< 0, 1, 1, 0 >, < 0, 1, 0, 1 >, < 0, 0, 1, 1 >}
In this paper %we assume homogeneous resources, namely each resource can apply
best practice defense equally for each of the targets, allowing all the possible
resource allocation schedules to be played


The mixed strategy D = 〈ds〉 of the
defender is a probability distribution over the different schedules, where ds is
the probability of playing a schedule s ∈ S.


. For example in a game with 120 strategies and rank 10,
around 3/4 become dominated when the rank is reduced to 6. By comparing
the equilibria found in a 10 targets, 2 system administrators game and its SVD
rank 2 abstraction (Table 2), we found that there is aperformance improvement
of more than 1000 times while the approximate solution only slightly deviates
from the precise solution

Considering the levels of discourse and quantification in game theory

philosophy of mathematics

"pure" mathematics

"applied" mathematics

computational mathematics

 


